\vspace*{5cm}
\begin{center}
\textbf{LỜI CẢM ƠN}
\end{center}
\hspace{6mm}Lời đầu tiên, tôi xin gửi lời cảm ơn trân trọng và sâu sắc tới các thầy cô thuộc Bộ môn Tin học - Khoa Toán tin trường Đại học Thăng Long, đã tận tâm truyền đạt những kiến thức quý báu trong quá trình học tập và thực hiện khóa luận này. \par
Đặc biệt, tôi xin gửi lời cảm ơn chân thành và sâu sắc tới  TS. Mai Thuý Nga và thầy Nguyễn Đức Thắng, người đã trực tiếp hướng dẫn tận tình và đóng góp những ý kiến quý báu. \par
Cuối cùng, tôi xin được gửi lời cảm ơn đến gia đình, người thân và bạn bè của tôi, những người đã ở bên, khuyến khích và động viên trong cuộc sống, học tập.\par
Trong quá trình thực hiện khóa luận của mình, tôi đã có gắng hết sức để tìm hiểu và hoàn thiện một cách tốt nhất. Nhưng với kiến thức và sự hiểu biết còn hạn chế, khóa luận này sẽ không tránh khỏi những thiếu sót, kính mong nhận được những góp ý của các thầy cô, các bạn và những người quan tâm đến khóa luận này.\\
[0.5cm]
\hspace*{5mm} \textbf{Tôi xin chân thành cảm ơn!}\\
[0.5cm]
\hspace*{11cm} \textit{Sinh viên thực hiện}\\
[1.5cm]
\hspace*{11.3cm} \textbf{Bùi Công Hoàng}


\newpage
\vspace*{5cm}
\begin{center}
\textbf{LỜI CAM ĐOAN}
\end{center}
\hspace{5mm} Tôi xin cam đoan đề tài tìm hiểu về mạng nơ-ron tích chập, ứng dụng cho bài toán nhận dạng biển báo giao thông được trình bày trong khóa luận này là do tôi thực hiện dưới sự hướng dẫn của TS. Mai Thuý Nga và thầy Nguyễn Đức Thắng.\par
Tất cả các bài báo, tài liệu, công cụ phần mềm của các tác giả khác sử dụng trong tài liệu đều được chỉ dẫn tường minh về nguồn và tác giả trong phần tài liệu tham khảo.\\
[1cm]
\hspace*{9cm} Hà Nội, ngày 1 tháng 1 năm 2019\\
\hspace*{10.5cm} \textit{Sinh viên thực hiện}\\
[2cm]
\hspace*{10.5cm} \textbf{Bùi Công Hoàng}\\
\thispagestyle{empty}

\newpage
\begin{center}
 \textbf{MỞ ĐẦU}
\end{center}
\par
\textit{Học sâu}, có tên gọi tiếng anh là \textit{deep learning}, là một thuật toán dựa trên một số ý tưởng từ não bộ tới việc tiếp thu nhiều tầng biểu đạt, cả cụ thể lẫn trừu tượng, qua đó làm rõ nghĩa của các loại dữ liệu. Deep Learning được ứng dụng trong nhận diện hình ảnh, nhận diện giọng nói, xử lý ngôn ngữ tự nhiên. Với dữ liệu khổng lồ hiện nay thì deep learning được ứng dụng vào rất nhiều các bài toán nhận dạng và cho thấy tính hiệu quả, độ chính xác cao so với các phương pháp truyền thống. \par
Những năm gần đây, ta đã chứng kiến được nhiều thành tựu vượt bậc trong ngành Thị giác máy tính (Computer Vision). Các hệ thống xử lý ảnh lớn như Facebook, Google hay Amazon đã đưa vào sản phẩm của mình những chức năng thông minh như nhận diện khuôn mặt người dùng, phát triển xe hơi tự lái hay drone giao hàng tự động. \par
Mạng nơ-ron tích chập (Convolutional Neural Network - CNN) là một trong những mô hình deep learning tiên tiến giúp chúng ta xây dựng được những hệ thống thông minh với độ chính xác cao như hiện nay. Trong khóa luận này, tôi đi vào nghiên cứu về mạng nơ-ron cũng như mạng nơ-ron tích chập, ý tưởng của mô hình CNN trong phân lớp ảnh (Image Classification), và áp dụng trong việc xây dựng hệ thống phân loại biển báo giao thông.\par
Nội dung khóa luận gồm 6 chương:
\begin{itemize}
\item[] \textbf{Chương 1: Tổng quan về bài toán nhận dạng biển báo giao thông.} Tại chương này tôi sẽ trình bày khái quát về bài toán nhận dạng biển báo và ứng dụng trong tương lai.
\item[] \textbf{Chương 2: Cở sở lý thuyết.} Ở chương này tôi đề cập một số kiến thức cơ bản để tiếp cận với cách hoạt động của mạng nơ-ron dễ dàng hơn.
\item[] \textbf{Chương 3: Mạng nơ-ron.} Chương này tôi sẽ trình bày cấu trúc, thành phần và cách hoạt động của mạng nơ-ron và mạng nơ-ron tích chập. Bên cạnh đó là một số vấn đề gặp phải khi xây dựng mô hình và phương pháp giải quyết.
\item[] \textbf{Chương 4: Object detection.} Chương này tôi xin trình bày một số thuật toán về object detection, hay còn có tên gọi là nhận diện vật thể.
\item[] \textbf{Chương 5: Ứng dụng mạng nơ-ron tích chập vào bài toán nhận dạng biển báo giao thông.} Tại chương này, tôi sẽ trình bài cách áp dụng mạng nơ-ron cho bài toán nhận dạng biển báo giao thông.
\item[] \textbf{Chương 6: Kết luận và hướng phát triển.} Chương này sẽ tóm tắt kết quả đạt được và các vấn đề khó khăn cũng như hướng phát triển trong tương lai.
\end{itemize}
